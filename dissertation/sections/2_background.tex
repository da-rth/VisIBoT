% Checked with Grammarly - 21/03/2021

\chapter{Background}

\section{The evolution of Botnets}

The earliest of botnets are believed to have originated from back-door concepts first seen in computer viruses discovered in 1999, namely the Sub7 Trojan and the Pretty Park Worm. Both viruses established the notion of using the IRC protocol to send execution commands, disguised as IRC messages, to remotely infected hosts through an IRC server. This discovery ultimately led to the propagation and coordination of the infected computer networks using the IRC protocol as a back-door, sparking the development of a new variant of malware used to cultivate and control large networks infected 'bot' devices, known as botnets. \citep{Ferguson2010}

According to \citet{Schiller2007}, one of the first IRC botnets is believed to be the Global Threat (GT) botnet. This worm-like malware manipulates the IRC communication protocol as a means of centralised botnet communication. Upon infecting a Windows computer system, the infected host would remotely connect to an IRC server, waiting to receive commands from a central Command \& Control (IRC) server whilst disguised as a seemingly legitimate IRC client. This centralised communication method through a central IRC server proved advantageous for botnet owners, as botnet traffic is concealed in plain-sight amongst legitimate IRC traffic. Additionally, through the propagation of IRC-based botnets, owners could use infected hosts to perform out a variety of commands, including port scanning, password sniffing, and DDoS attacks. This technique led to the development of several other notorious botnets, such as the MaXiTE IRC botnet. However, authorities were eventually able to detect such IRC botnets through intrusion detection, binary analysis, and network traffic monitoring. As infected bots had to know the IP address and port of the IRC communication server, authorities could shut down offending IRC servers following identification through inspection of sources such as malware binaries and network traffic.

Given the various shortcomings of IRC-based botnet communication, botnet developers began to employ other protocols and application layer services as a means of communication, such as the now extremely popular HyperText Transfer Protocol (HTTP) and Domain Name Service (DNS) protocols. In particular, botnets, phishing websites, and other malware distributors began to exploit the DNS protocol by using 'Fast Flux' domains. A Fast Flux DNS enables botnets to "hide behind an ever-changing network of compromised hosts, ultimately acting as proxies" \citep{Morton2017}. This presents challenges for botnet mitigation as Command \& Control servers masked behind such botnet traffic are significantly more challenging to identify. According to \citet{Morton2017}, the Storm botnet is one of the first botnets to mask remote servers behind using Fast Flux DNS. Deployed in 2007, the Storm botnet is also known as one of the first de-centralised botnets through the transmission of botnet communication through standard Peer-to-Peer protocols. Early iterations of the Storm botnet have used OVERNET; a Peer-to-Peer distributed hash table (DHT) routing protocol, which bots used to find other bots and receive commands through a Peer-to-Peer network. \citep{Holz2008} This transition from centralised to de-centralised communication is said to have marked a significant point in the evolution of botnets, as botnets can be controlled without the concern of deployment and obfuscation factors presented when using a central C2 server.

Similarly, \citet{ElDefrawy2007} discuss how malicious actors could exploit standard P2P protocols such as BitTorrent to allow for de-centralised botnets and the coordination of large-scale DDoS attacks via peer-to-peer networks. As all bit-torrent clients rely on interacting with trackers, one possible exploit includes disguising a target DDoS victim as a bit-torrent tracker, generating a highly distributed attack using connected P2P nodes. An alternative exploit identified by \citet{ElDefrawy2007} is that BitTorrent Distributed Hash Tables (DHTs) can be used to coordinate botnet attacks on a victim IP address through botnet owners pinging a DHT with a target IP address, which bots will then target. 

Published anonymously due to potentially unethical and potentially illegal experimentation, the \textit{Internet Census 2012} \citep{Cencus2012} is perhaps one of the first papers to discuss the insecurities of IoT devices within the context of botnet propagation. In this paper, it was found that a significant number of IoT devices were vulnerable to unauthorised remote access by attempting to log in using common factory-set usernames and passwords created by manufacturers. Following this discovery, the author developed Carna - a distributed port-scanning framework used for botnet propagation. By infecting select vulnerable devices with a port-scanning binary, infected devices would scan other vulnerable machines over the telnet protocol using various scanning tools and methods. This activity ultimately led to the identification of 1.3 billion in-use IP addresses. Additionally, the Carna botnet expanded to over 420,000 vulnerable devices, marking a new and highly effective propagation technique for botnet developers to employ.


% =========================================================================================


\section{An overview of IoT Botnets}

Shortly after the anonymous release of the findings presented in the Internet Census 2012, a new variation of botnets began to surface, targeting a diverse selection of operating systems and CPU architectures common to IoT devices. Aidra, also known as Lightaidra, \citep{AidraSource} was discovered in January of 2012 by ATMA.ES. \citep{ATMA2012} Security researchers observed a substantially high number of telnet attacks were identified to be originating from various ARM-based IoT devices. \citep{NJCCIC2016} It was later found that despite the botnet initially targeting ARM-based IoT devices, the malware binaries used for propagation could also be compiled to several other IoT architectures, including MIPS, MIPSel, PPC, x86/x86-64, and SuperH. Aidra could propagate similarly to that of the Carna botnet, by first port-scanning for vulnerable IoT devices using the telnet protocol. Once identified, the botnet attempts to infect the device through brute-forcing known login credentials and sending malicious requests containing command-line injection code, which downloads and executes a malware binary. Like previous botnets, the Aidra botnet uses the IRC communication protocol as a back-door, such that infected devices connect to a central IRC C2 server and await new commands from the botnet owner. However, as several IoT devices operate using different CPU architectures, the command-line injection code attempts to download and execute all architecture variations of the Aidra malware binary from a remotely hosted Payload server.

Bashlite, also known as Gafgyt, LizardStresser, Lizkebab, PinkSlip, Qbot and Torlus, \citep{NHS2018} was identified in 2014 as a new botnet variant that targets specific IoT devices. The first iteration of the Bashlite botnet malware initially targeted various router devices running outdated versions of the BusyBox software to exploit the GNU Bash vulnerability known as ShellShock. \citep{Kovacs2014} As the versions of GNU Bash affected by this vulnerability inadvertently allow for the processing of trailing strings after function definitions in the values of environment variables, attackers could use this exploit to execute arbitrary code through the likes of command-line injection. \citep{CVE20146271} All scanning activity is controlled by the C2 server through sending commands to infected bots. \citet{Marzano2018} Upon receiving a scan command, a bot will begin to randomly scan SSH and Telnet ports for vulnerable devices and report potential targets to a loader server. The loader server will attempt to login to the device via SSH or Telnet protocols through brute-forcing commonly used passwords. Once the server gains unauthorised access, it will download, execute, and infect the device with a botnet malware binary. The IP address and port of the C2 server are often hard-coded within the binary so that a handshake between the bot and C2 server can be performed once the device is infected. Using Bashlite's command distribution functionality, a bot owner can perform various attacks, including DDoS, TCP Flood, and SYN Flood attacks. These commands are typically un-encrypted and sent from a C2 server as plain-text. Additional functionality includes distributing binary updates using management commands, collecting statistical information on a botnet using query commands, and stopping on-going attacks using interrupt commands.

First identified in 2016 by \citet{Malwaremustdie2016}, the Mirai botnet improves upon Bashlite in several respects. \citep{Marzano2018} The most considerable change proposed by the Mirai botnet was the brute-forcing of a broader range of potential target IoT devices, including DVRs, routers, and IP Cameras. Due to the infection of non-conventional devices, it is proposed that the reason for the lack of detection was due to the difficulty of detecting and extracting malware samples from infected IoT devices. \citep{Malwaremustdie2016}. Honeypot servers could not detect or collect Mirai binaries as particular ports, vulnerabilities, and device architectures common to IoT devices were being used for propagation, allowing for botnet growth to go unnoticed for a considerable amount of time. In a study by \citep{Marzano2018}, several differences between Mirai and Bashlite were observed. Like that of Bashlite, Mirai also employs telnet port-scanning and login credential brute-forcing techniques for botnet propagation. However, unlike Bashlite, password dictionaries comprising of default credentials of a wide range of IoT devices were used.

By targeting a more comprehensive range of IoT devices, the Mirai botnet can achieve better global distribution, as devices such as IP Cameras and DVRs are widely used across various regions. Unlike Bashlite, Mirai C2 servers are limited to the distribution of attack commands. All port-scanning activity is automatically performed by infected bots, emitting the distribution of C2 port-scanning commands. When a device is infected with Mirai, the binary will execute code that blocks access to ports commonly used for infection, such as 23 and 2323, and actively removes competitor botnet malware to prevent future re-infection. Mirai also utilises DNS-based botnet communication, removing the use of hard-coded IP address and ports common to Bashlite variants. DNS allows for quick and easy hot-swapping of C2 server IP addresses using DNS name records. Should a C2 server be taken down by authorities, the botnet owner can change the IP address. As mirai botnets communicate with C2 servers using a domain instead of an IP address, bots can continue to be controlled even if the C2 server is compromised. That is, the IP address associated with the DNS record can be changed on the spot. DNS also supports various means of traffic encryption, enabling botnet owners to communicate via encrypted messages between bots through DNS over TLS or HTTPS, making packet analysis and traffic monitoring significantly more challenging. 

Following the public release of the Mirai botnet in 2016, \citet{Antonakakis2017} propose that the availability of the Mirai botnet source-code has sparked a new generation of Mirai-based botnet variants, marking a new point in the evolutionary scale of botnet history. Hajime, a recent variant of Mirai, is evidence of such evolution. Unlike Mirai, the Hajime botnet has adopted a de-centralised strategy to botnet communication through current Peer-to-Peer technologies. This strategy eliminates the need for a centralised C2 server, as peer-to-peer botnets can receive commands and files from each other. As there is no central point of communication, the 'serverless nature of P2P botnets makes identification and mitigation significantly more challenging than centralised botnet infrastructures. \citet{Herwig2019} found that the Hajime botnet employs P2P botnet communication through a Distributed Hash Table (DHT) overlay network available via the BitTorrent P2P protocol. Using a DHT overlay network allows for the distribution of software updates and attack commands across all connected nodes within a given P2P botnet network. Once a device is infected with the Hajime payload, it will join a DHT through a known peer and download content or receive commands from connected peer-nodes marked as 'seeders'. As the BitTorrent DHT network is publicly available and actively used by legitimate peers, bots often interact with legitimate peers to introduce noise during any potential network analysis. One significant characteristic of the de-centralised Hajime botnet is that infected bots are used to self-propagate by individually downloading, hosting, and attacking other IoT devices using Hajime malware binaries exclusive to the architecture of the bot.

Despite the feature-set offered by the Hajime Peer-to-Peer IoT botnet, experts have observed a recent influx in de-centralised botnet traffic coming from a new Mirai botnet variant known as Mozi. \citep{Netlab2019} In the same way as Hajime, Mozi utilises a de-centralised Peer-to-Peer structure using Distributed Hash Tables. However, the Mozi botnet uses it's own DHT protocol for propagation, utilising various public nodes from popular P2P services as bootstraps for guiding infected bots to the botnet DHT. These nodes are specified in a config file downloaded following successful infection through telnet brute-forcing and command-line injection. Similarly to Hajime, the Mozi botnet also employs a unique strategy for botnet propagation, as infected devices actively self-host malware binaries. Once infected, the bot will download Mozi malware binaries and DHT configuration files and set up an HTTP server assigned to a randomly selected port. The bot will attempt to infect other devices using the malware hosted on its own server, alleviating the need for payload or loader servers commonly used by Mirai botnets. The port number will change over time, ensuring binaries are no-longer retrievable from previously used ports after a given time-frame. \citet{Netlab2019} observed that the Mozi botnet uses a specific identifier starting with \texttt{1:v4:} which is used to distinguish Mozi bots from legitimate P2P traffic. This identifier is followed by a flag consisting of 4 bytes. The first byte is randomly generated, the second is hard-coded in the Mozi config file, and the 3rd and 4th bytes are calculated using an algorithm found in Mozi's source code.


% =========================================================================================


\section{Defence against IoT Botnets}

As the rise of IoT vulnerabilities continues to increase, \citet{Bertino2017} argue that IoT botnet research must remain a high priority amongst cyber-security specialists. The propagation and utilisation of such botnets have proven to pose drastic effects on even the most secured internet services. \citet{Bertino2017} state that there are eight categories of vulnerabilities common to IoT devices that must be resolved to mitigate attacks coordinated by malicious actors using IoT botnet malware. This list includes several vulnerabilities explicitly targeted by the Mirai botnet: the use of insecure web interfaces and control panels, known default login credentials, vulnerable web services, and insecure operating systems or firmware. The authors argue that monitor both the network traffic and device behaviour of IoT devices should remain a central concern for mitigating botnet activity, as even secured devices can still be infected with malware. Through analysing network traffic for anomalies using Network Behavioural Analysis (NBA), potential botnet activity can be investigated and countered accordingly. It is suggested that various metrics can be used to identify such anomalies, including monitoring network bandwidth and the usage of network protocols.

In the past, real-time botnet traffic analysis was challenging to coordinate. Thus, systems often relied on post-attack data such as network logs for identifying botnets. This reactive approach for botnet identification was not ideal as attacks need to take place first. \citet{Lividas2006} proposed a solution for preemptively identifying botnets without relying on data collected from attacks that previously took place. Using machine learning classifiers, \citet{Lividas2006} suggested that botnet traffic could be identified through IRC network traffic flow analysis. However, as both technology and the internet has advanced, the Peer-to-Peer and Hypertext Transfer Protocol (HTTP) protocols are now some of the most common means of communication for botnet traffic, gradually rendering the use of IRC-based botnets obsolete. \citet{Moon2012} propose botnet identification through purposeful execution and dynamic/network analysis of malware binaries using an advanced honeypot system. Integrated with the back-end infrastructure of KORNET, the largest ISP service in Korea, this system showed a prediction accuracy of approximately 40\% for future blacklisted botnet traffic. Despite the valuable information obtained through static and dynamic analysis of malware binaries, this system's resulting low accuracy was likely due to un-optimal malware sample extraction from trivial data-sources.

Despite the use of conventional honeypots showing some success in the findings presented by \citet{Moon2012}, it is suggested by \citep{Malwaremustdie2016} that typical honeypot infrastructures are not able to detect IoT botnet activity effectively. Botnets such as Mirai target particular devices based on CPU architecture and exposed ports. \citet{Antonakakis2017} propose that IoT botnet activity can be captured by configuring honeypots to mimic vulnerable IoT devices' characteristics by configuring them to use BusyBox and respond to requests with IoT-like headers. This method proved somewhat effective, logging connection attempts from over 80,000 foreign IP addresses over approximately five months and collecting a total of 151 unique malware binaries. Combining this honeypot information with various other sources, \citet{Antonakakis2017} identified different trends common to Mirai botnets, including several infection strategies, commonly targeted ports, and an analysis of high profile attacks. The study also investigates geographic patterns using the MaxMind GeoIP 2 database for collecting IP address geolocation information. This information provided an insight into which countries are hot-spots for Mirai device infection and how detected botnets are globally distributed. 

Similarly, \citet{PaPa2016} also noted that current honeypot systems proved insignificant for extracting IoT malware information, and thus presented \textit{IoTPOT}, an improved IoT honeypot system with sandbox integration. By imitating various IoT devices, the proposed honeypot system offers a novel approach to collecting IoT-specific malware samples from malicious actors by utilising sandbox environments running embedded software. This proved considerably more effective non-IoT honeypot systems, such as Honeyd \citep{Honeyd2008}, as the virtual sandboxes could be expanded to support various IoT architectures, including ARM, MIPS, and PPC. Additionally, IoTPOT logs all telnet-based intrusion attempts made by visiting IP addresses, allowing for the extraction and collection of malware samples. However, this implementation yielded a potential risk of honeypot instances becoming infected following successful intrusion attempts, in which case the sandbox environments would have to be completely reset. Some preventive measures were put in-place to attenuate potential side-effects of infected hosts: outgoing Denial of Service (DoS) traffic was actively blocked, and all other traffic was rate-limited by the host network. Following successful infection, IoTPot actively redirects all port-scanning activity to a dummy server, enabling the monitoring and observation of botnet propagation activity. 

\citet{Bastos2019} build upon the concept of preemptive botnet detection through purposeful execution of malware binaries by collecting samples via 47 low-interactivity honeypot servers distributed across 15 Brazilian states. Similarly to \citet{Moon2012}, this implementation infers candidate C2 servers of Bashlite and Mirai botnets using four heuristics based on a combination of static and dynamic analysis techniques. Such heuristics include analysing hard-coded IP addresses in binary strings, contacted DNS names, payload server URLs, and port-specific connections uncommon to the scanning process. Once identified, candidate C2 servers are validated by executing variant-specific handshake procedures between botnets and infected hosts within the IoT sandboxing environment \textit{Detux Sandbox} \citep{Detux2016}. This environment supports binaries compiled to run on common IoT device architectures, including ARM, MIPS and x86. Binaries were executed for only 90 seconds, and common scanning ports such as the telnet ports 23 and 2323 were blocked to prevent significantly malicious activity from occurring during analysis. This system successfully identified and validated C2 servers for 62\% of binaries. \citet{Ceron2019} builds upon the framework suggested by \citet{Bastos2019} by focusing on improvements towards the dynamic and network analysis procedures previously suggested. It is noted that blocking specific ports could potentially lead to valuable network traffic from being blocked, assuming that some C2 servers communicate over commonly used ports such as 23 or 2323. Alternatively, \citet{Ceron2019} suggest using an adaptive network layer, similar to that of the strategies employed by \citet{PaPa2016}, that monitors, redirects and drops malicious packets. Redirecting packets through a Software Defined Network (SDN) ensures that no traffic data is lost due to the blocking of ports. Additionally, all outgoing traffic following a successful handshake with a C2 server can be redirected to an impersonator C2 server for improved behavioural analysis of botnet activity.

Alternatively, \citet{Dwyer2019} propose the identification of IoT botnets through DNS-based analysis using machine learning. The data-set used for evaluation was collected using a distributed honeypot collection system capable of detecting Mirai-like TCP sequences. Mirai-like activity detection was implemented by comparing the TCP sequence fingerprint of an incoming packet with the IP address that was being targeted. If the sequence matches the IP address, the packet is deemed to be 'Mirai-like'. By distributing honeypots across the United States, Russia, and Brazil, the honeypot system collected a total of over 800,000 Mirai-like probes over approximately two years, 25\% of which were successfully reverse-queried to obtain the DNS names of all logged IP addresses. The DNS records collective were tested against a set of legitimate DNS records collected from the Majestic Million Rankings list. A total of 6 features were extracted from each DNS record, including the Time To Live of the DNS record, the diversity of Autonomous Systems across all IP addresses associated with a DNS record, the Spatial distribution of IP addresses, the vowel density the domain name, and the Shannon entropy (randomness) of the domain name. Dwyer et al. found that the vowel density and Sharron entropy features gave the best results, yielding a detection accuracy rate of over 99\% when using the Random Forest classification model.

To identify de-centralised, Peer-to-Peer IoT botnets, \citet{Wang2017} propose a detection model for identifying malicious data-streams amongst legitimate P2P traffic. Similar to previous propositions, the detection system actively identifies peer-to-peer botnets within the botnet propagation stage, such that botnet activity can be identified and monitored before attacks occur. The procedure involves first filtering out legitimate P2P traffic using characteristics such as packet size and monitoring the flow of packets between nodes of a peer-to-peer network in real-time. \citet{Li2019} builds upon the concept of data-flow analysis for P2P botnet detection by proposing a detection system using distributed intelligence sharing. Potentially malicious P2P traffic is collected from distributed communal clusters, which have raised potential threat alerts with the central hub. Upon receiving the network traffic flows of possible bot devices from community-based clusters, these flows are analysed using an algorithm based on the hidden Markov model.

Alternatively, \citet{Herwig2019} propose that a similar honeypot detection system similar to that of \citep{Bastos2019, Ceron2019} can be applied to Hajime botnets. Hajime botnet propagation and attack activity can be actively monitored through performing handshakes with Hajime botnet nodes through performing handshakes with infected P2P nodes. Once an authenticated handshake is established, \citet{Herwig2019} could gain access to and actively monitor the Distributed Hash Tables of several Hajime botnets, logging a total of over 10 million public keys. This is accomplished through computing the same identifier keys as identified bots during malware analysis and exhaustively looking up the keys in the Distributed Hash Tables commonly used by Hajime. However, as this method focuses on Distributed Hash Table patterns and handshake procedures unique to the Hajime botnet, the proposed strategy cannot be directly applied to other P2P IoT botnets, such as Mozi. As identified by \citet{Netlab2019}, the Mozi botnet uses it's own Distributed Hash Table protocol and has an extensive signature validation process strictly enforced between all botnet nodes. If other nodes cannot validate a Mozi node's signature, it will be excluded entirely from the botnet. The use of a valid signature may allow access to the Mozi DHT, though, Mozi botnet activity can still be identified from within a host system by analysing system processes and monitoring HTTP and DHT network connections. 