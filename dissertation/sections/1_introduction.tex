% Proof read

\chapter{Introduction}

% reset page numbering. Don't remove this!
\pagenumbering{arabic} 

On the 21st of October, 2016, a leading Domain Name Service, known as DynDNS, was targeted by Hacktivists in one of the most significant Distributed Denial of Service (DDoS) attacks recorded to date. Lasting several hours, the outages following the attack directly impacted many services, including Cloudflare, Amazon, Twitter, Netflix, and GitHub. \citep{Maunder2016} Unlike those seen before, the DynDNS DDoS attack was performed by thousands of Internet of Things (IoT) devices, coordinated by a botnet known as Mirai. A botnet is a network of private computers that have been infected with malicious software. All computer systems within the network, i.e. bots, are controlled as a group to perform malicious tasks, such as spam distribution or denial of service attacks. \citep{Lexico2021} It is estimated that the Mirai botnet infected over 500,000 IoT devices before the attack. Its fast propagation allowed over 100,000 infected IoT devices to participate in the attack against DynDNS, yielding a maximum attack magnitude of 1.2 Tbps. \citep{Threatpost2016} Contrary to conventional botnets, which often operate using infected computer desktops, the Mirai botnet exclusively targets vulnerable IoT devices, including unprotected IP/CCTV cameras, baby monitors, and routers.

The Internet of Things (IoT) describes the intercommunication of embedded systems, aptly described as 'things', which actively communicate and exchange data with other devices. \citep{Oracle2021} Arm, a leading provider in processors for embedded systems, characterises IoT devices as "pieces of hardware, such as sensors, actuators, gadgets, appliances, or machines, that are programmed for certain applications and can transmit data over the internet or other networks." \citep{Arm2021} Industrial IoT devices are often embedded within other mobile devices, environmental sensors, and industrial or medical equipment. However, commercial IoT devices are also commonly used in home and working environments, including Digital Video Recorders (DVR), IP-Cameras, Internet Routers, Smart Home Security Systems, and Voice Assistants. Such devices have become increasingly popular because of increased connectivity, low cost and power consumption, and integration with cloud computing platforms, artificial intelligence, and machine learning. \citep{Oracle2021} \citet{Lueth2020} approximates that 54\% of the 21.7 billion devices connected to the internet in 2020 were that of IoT devices. Due to the rampant manufacturing of inexpensive IoT devices, it is estimated that, by the end of the year 2025, over 30 billion active IoT devices will be connected, amounting to almost four IoT devices per person.

\section{Motivation}

Following a recent surge in demand for IoT devices, several manufacturers have expanded to compete in a new and highly competitive market. Due to a lack of consideration for security and update-ability, the rapid production of IoT devices has resulted in the release of millions of vulnerable IoT devices. The Open Web Application Security Project (OWASP) reported that the top 10 most common IoT Vulnerabilities include weak, guessable, or hard-coded passwords. Additionally, IoT devices often have insecure network services or ecosystem interfaces, a lack of secure updating systems and device management, and operate using insecure data transfer protocols. \citep{Misra2021} Known as Common Vulnerability Exploits (CVE), these vulnerabilities are frequently exploited by bad actors, such as botnet owners, to cultivate and propagate IoT-based botnets. As many targeted IoT devices are often un-managed on an administrative level, intrusions often go unnoticed by the owner, allowing the device to participate in distributed cyber-attacks surreptitiously. 

In light of the Mirai botnet source-code release in 2016, many anticipated that the internet would soon be flooded with Mirai-like botnets by highly vulnerable and easily hackable IoT devices. \citep{Krebs2016} As IoT device popularity increases, so too will the number of vulnerabilities. The evolution of botnets and web application frameworks has also resulted in an influx of online DDoSing services, allowing individuals to hire botnets for orchestrating attacks temporarily. Depending on the botnet's size, renting for DDoS attacks can cost up to several thousand dollars per day, enabling significant monetary gain for botnet owners. \citep{Putman2018} The cost of identifying, mitigating, and recovering from botnet DDoS attacks is believed to be in the millions. DDoS attacks have a detrimental impact on both targeted services and their dependents and have been known to cause large-scale disruptions capable of affecting millions of users. Government agencies, commercial services, and Cyber Threat Intelligence agencies have great interest in identifying and alleviating such botnets and seek to employ large-scale botnet detection systems to allow for preemptive botnet detection. Many botnets operate using centralised Command \& Control (C2) servers used to distribute commands amongst bots, thus, authorities have found success in mitigating botnets through identifying and shutting down such servers. By taking out the C2 server of a botnet, the central point of communication is interrupted and infected bots are rendered useless. However, botnet developers are now employing de-centralised communication protocols. By removing central points of contact through a employing a peer-based communication strategy, perpetrators can significantly reduce the likelihood of interference from authorities. 

\section{Aims and Objectives}

Current botnet detection systems are often highly theoretical, target very specific communication protocols, and are often difficult to apply within real-time, automated, and globally distributed context. To alleviate these challenges, I propose VisiBot: a modular, automated, and distributable botnet detection and visualisation tool. In collaboration with Bad Packets, \citep{BadPackets} the proposed system aims to identify and analyse internationally distributed IoT botnets by profiling and analysing potentially malicious network packets collected from a globally distributed IoT honeypot network. Using automated and distributable technologies, the VisiBot Processing System achieves real-time, extensible botnet detection through:

\begin{itemize}
    \item Automatic honeypot packet collection, classification, and malware sample extraction
    \item Automated static and dynamic analysis of malware samples using the LiSa sandbox \citep{LiSa}
    \item Heuristic-based detection of candidate IoT botnet C2 Servers and P2P botnets
    \item Supplementary monitoring of geographic density and Autonomous System (AS) interactions of identified botnets, and web-based geographic clustering visualisation
\end{itemize}

\section{Summary}

Following a 35-day data-collection and evaluation period, the VisiBot Processing System classified packets originating from a total of 61,293 unique public IP addresses. A total of 9,923 payload URLs were identified, followed by the successful extraction and analysis 1,654 of malware samples. Additionally, a total of 1,303 candidate C2 servers and 6,876 Peer-to-Peer botnet nodes were identified using four identification heuristics. Throughout this paper, several topics will be discussed, including an overview of relevant background research, an analysis of the problem at hand, an explanation of the design overview and implementation, and an evaluation of results collected and the system's effectiveness whole.