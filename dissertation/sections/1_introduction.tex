\chapter{Introduction}

% reset page numbering. Don't remove this!
\pagenumbering{arabic} 

On the 21st of October, 2016, a leading Domain Name Service, known as DynDNS, was targeted by Hacktivists in one of the most significant Distributed Denial of Service (DDoS) attacks recorded to date. Lasting several hours, the outages caused by this attack directly impacted a multitude of services, including Cloudflare, Amazon, Twitter, Netflix, and GitHub. \citep{Maunder2016} Unlike attacks seen before, the DynDNS DDoS attack was performed by thousands of Internet of Things (IoT) devices, coordinated by a botnet known as Mirai. A botnet is a network of private computers that have been infected with malicious software. All computer systems within the network, i.e. bots, are controlled as a group to perform malicious tasks, such as spam distribution or denial of service attacks. \citep{Lexico2021} It is estimated that the Mirai botnet infected over 500,000 IoT devices before the attack. The Mirai botnet's fast propagation allowed over 100,000 infected IoT devices to participate in the attack against DynDNS, yielding a maximum attack magnitude of 1.2 Tbps. \citep{Threatpost2016} Contrary to conventional botnets which operate on infected computer systems, such as desktops and laptops, the Mirai botnet exclusively targeted vulnerable IoT devices, including unprotected IP/CCTV cameras and internet routers.

The Internet of Things (IoT) describes the intercommunication of embedded systems, aptly described as 'things', which actively communicate and exchange data with other devices. \citep{Oracle2021} Arm, a leading provider in processors for embedded systems, characterises IoT devices as "pieces of hardware, such as sensors, actuators, gadgets, appliances, or machines, that are programmed for certain applications and can transmit data over the internet or other networks." \citep{Arm2021} Industrial IoT devices are often embedded within other mobile devices, environmental sensors, and industrial or medical equipment. However, commercial IoT devices are also commonly used in home and working environments, including Digital Video Recorders (DVR), IP-Cameras, Internet Routers, Smart Home Security Systems, and Voice Assistants. Such devices have become increasingly popular because of increased connectivity, low cost and power consumption, and integration with cloud computing platforms, artificial intelligence, and machine learning. \citep{Oracle2021} \citet{Lueth2020} approximates that 54\% of the 21.7 billion devices connected to the internet in 2020 were that of IoT devices. Due to the rampant manufacturing of inexpensive IoT devices, it is estimated that, by the end of the year 2025, over 30 billion active IoT devices will be connected, amounting to almost four IoT devices per person.

Following a recent surge in popularity for IoT devices, several manufacturers have expanded to compete in a new and highly competitive market to produce IoT devices. Due to a lack of consideration for security and update-ability, IoT devices' rapid production has resulted in the release of millions of vulnerable IoT devices. The Open Web Application Security Project (OWASP) reported that the top 10 most common IoT Vulnerabilities include weak, guessable, or hard-coded passwords. Additionally, IoT devices often have insecure network services or ecosystem interfaces, a lack of secure updating systems and device management, and operate using insecure data transfer protocols. \citep{Misra2021} Known as Common Vulnerability Exploits (CVE), these vulnerabilities are frequently exploited by bad actors, such as botnet owners, to cultivate and propagate IoT-based botnets. As many targeted IoT devices are often un-managed on an administrative level, intrusions often go unnoticed by the owner, allowing the device to participate in distributed cyber-attacks surreptitiously. 

In light of the Mirai botnet source-code release in 2016, many anticipated that the internet would soon be flooded with Mirai-like botnets by highly vulnerable and easily hackable IoT devices. \citep{Krebs2016} The evolution of botnets and web application frameworks has also resulted in an influx of online DDoSing services, allowing individuals to hire botnets for orchestrating attacks temporarily. Depending on the botnet's size, renting for DDoS attacks can cost up to several thousand dollars per day, enabling significant monetary gain for botnet owners. \citep{Putman2018} Adversely, the cost of identifying, mitigating, and recovering from botnet DDoS attacks is believed to be in the millions. As DDoS attacks have a detrimental impact on targeted services and external dependants, significant effort is being placed on the identification and alleviation of botnets by being placed on government entities' likes, Internet Service Providers (ISP), commercial businesses, and Cyber Threat Intelligence agencies. Botnets are conventionally controlled using centralised communication strategies, where connected all bots communicate with a central Command \& Control (C2) server. As a bot must first receive a command containing the target before participating in an attack, authorities have focused on identifying, pursuing, and shutting down Command \& Control servers, rendering corresponding botnets useless. However, botnet developers have sought alternative communication protocols to alleviate any interference with authorities. In particular, IoT botnets have continued to evolve in recent years, implementing DNS and Peer-to-Peer (de-centralised) communication protocols to limit authorities' interference. The constant evolutionary nature of botnets has proven to be a continual challenging for parties concerned with preventing cyber-attacks. Attackers are constantly seeking to exploit new vulnerabilities, target new devices, and employ various communication protocols. As the number of IoT devices continues to grow, so too will the number of vulnerabilities in IoT devices, presenting a bountiful range of devices that IoT botnets can recruit. Additionally, as botnets continue to evolve and utilise complex communication systems such as Peer-to-Peer, detecting the presence of such botnets during their propagation stage will become considerably more difficult.

In consideration of the current challenge of automatically identifying evolving IoT botnets, I propose VisiBot. VisiBot is an automated IoT botnet detection and visualisation tool. The VisiBot Processing System is a framework used by VisiBot, which employs a globally distributed honeypot network to conduct automatic classification, malware extraction, sandboxing and network analysis of all potentially malicious packets. Using an automated Linux Malware Analysis tool known as LiSa, \citep{LiSa} the VisiBot Processing System collects a wide array of static, dynamic, and network-based information through the purposeful execution of malware samples extracted from honeypot intrusion attempts. By analysing the traffic generated by malware samples, botnets can be identified during their propagation stage before attacks. VisiBot explicitly targets IoT-based botnets, using a sandboxing system that supports up to 5 IoT CPU architectures, including i386, ARM, and MIPS. By applying four novel heuristics to the static and network information collected, VisiBot facilitates the automatic detection of candidate IoT botnet Command \& Control servers and Peer-to-Peer botnet nodes. A real-time interactive Web Application was also built to visualise the geographic density and distribution of currently propagating IoT botnets. Using a combination of heterogeneous data-sources, a real-time representation of botnet activity is shown through the clustering of all geographic coordinates of all IP addresses classified by the VisiBot Detection System within a 24-hour window.

Following a 35-day data-collection and evaluation period, the VisiBot Processing System classified packets originating from a total of 61,293 unique public IP addresses. A total of 9,923 payload URLs were identified, followed by the successful extraction and analysis 1,654 of malware samples. Using four identification heuristics, a total of 1,303 candidate Command \& Control servers and 6,876 Peer-to-Peer botnet nodes were identified. Throughout this paper, several topics will be discussed, including an overview of relevant background research, an analysis of the problem at hand, an explanation of the design overview and implementation, and an evaluation of results collected and the system's effectiveness whole.