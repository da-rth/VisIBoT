% Checked with Grammarly - 21/03/2021

\chapter{Conclusion}

The combined process of automated traffic classification, payload extraction, and sandbox-based malware analysis has proved highly effective for identifying and visualising IoT botnet traffic. VisiBot has shown to be robust, extensible, and capable of scaling demand by employing a globally distributed honeypot network and Message Broker-based analysis system. The distribution of honeypot and malware analysis tasks across multiple workers allows for real-time and parallelised packet classification, payload URL extraction, and malware analysis. The collection of botnet entity IP address meta-data, such as geographic location, Common Vulnerability Exploits, and Autonomous System history, provides sufficient knowledge for visualising and characterising current botnet behaviours and mannerisms, including the ubiquity and density of globally distributed botnets, leading countries of C2 and P2P botnet activity, and ASN trading patterns associated botnet IP addresses.

When applied to static, dynamic and network analysis of collected malware binaries, the proposed heuristics permit the identification of centralised and de-centralised botnet traffic using automated and distributed techniques. Through the profiling of commonly observed characteristics amongst both centralised and de-centralised botnets, heuristic analysis allows for a computationally inexpensive solution for detecting potential Command \& Control server and Peer-to-Peer botnet activity. The extraction capabilities of VisiBot have also shown to be effective in extracting payload URLs obfuscated within remote code execution attempts. However, as botnets evolve, so too will the obfuscation techniques employed. In its current configuration, the VisiBot Processing System proved capable of identifying and analysing traffic generated by various centralised and de-centralised botnet variants, including a recent IoT variant, the Mozi Peer-to-Peer botnet. 

In summary, the VisiBot Processing System presents an effective solution for automated identification of centralised and de-centralised IoT botnets through real-time parallelised extraction, execution, and heuristic analysis of honeypot malware samples. In doing so, the VisiBot Processing System successfully classified over 58,010 unique IP addresses, allowing for the identification, examination, and visualisation of 1,303 candidate Command \& Control servers and 6,876 Peer-to-Peer nodes associated with 1,654 malware samples over a 35-day evaluation period.
\section{Future Work}

Current capabilities of VisiBot allow for real-time candidate C2 server and Peer-to-Peer node detection. However, the proposed system does not validate C2 servers and P2P Nodes as existing VisiBot heuristics only identify characteristics that resemble C2 or Peer-to-Peer botnet traffic. Such candidates need to be validated to determine if they are genuinely malicious. Heuristic analysis carried out by VisiBot can be updated to include validation steps that perform various botnet handshakes on detected candidate C2 servers and Peer-to-Peer nodes to identify if a given endpoint is malicious. As described by \citet{Bastos2019}, IoT Command \& Control servers for variants such as Mirai and Bashlite can be validated through reverse engineering a malware sample, analysing its handshaking process, and performing and interpreting the handshake with a candidate C2 server. This process is equally necessary for Peer-to-Peer botnet detection, as botnets such as Mozi purposefully attempt to obfuscate Peer-to-Peer botnet activity through interacting with legitimate peers via publicly hosted Distributed Hash Tables. \citep{Netlab2019} As proposed by \citet{Herwig2019}, DHT protocols, used by the likes of Mozi or Hajime, can be monitored, and corresponding peer nodes validated through the successful invocation of a handshake.

Despite the proposed identification heuristics proving effective in candidate C2 and P2P detection, some heuristics proved less applicable than others throughout the evaluation period. Namely, one of the heuristics based on hard-coded IP addresses of malware binaries proved challenging to utilise. The VisiBot Processing System had difficulties unpacking several malware binaries packed with tools such as UPX \citep{UPX}, leading to the extraction of obfuscated strings during static analysis. Through implementing a robust malware unpacking procedure that supports various packing methods, crucial strings, such as hard-coded IP addresses used in the heuristic analysis, can be successfully extracted. By doing so, heuristics based on static analysis features, such as strings, will not be hindered by common obfuscation techniques. The application of additional identification heuristics may also allow for increased C2/P2P detection-rate and accuracy, using different data sources collected during static, dynamic and network analysis, such as captured packet files (pcaps) and machine logs.

Lastly, a considerable limitation of the VisiBot Processing System is that it is limited to only retrieving honeypot results from the Bad Packets \citep{BadPackets} honeypot service on an hourly basis. The hourly period between queries significantly limits the number of malware samples extracted from remote code execution attempts conducted by malicious bots from botnets such as Mozi. As such bots self-propagate using randomly expiring payload URLs, there is little time to collect the binary before the web-server's network port hosting the binary changes. A possible solution may be to employ a stream-based honeypot collection system that allows packets to be processed as soon as the honeypot network detects them. Processing such packets in truly real-time will allow for the immediate extraction of expiring payload URLs, which, as observed, are becoming increasingly more common.