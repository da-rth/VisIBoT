% Checked with Grammarly - 21/03/2021

\chapter{Problem Analysis}

The concept of preemptive botnet detection proposed by \citet{Moon2012} proved significant in detecting botnets, as the purposeful execution and analysis of malware enabled the collection and analysis of information crucial to the identification and mitigation of botnets during propagation periods. This system allows for the automatic collection and analysis of binaries using two dedicated processing modules. However, several issues are presented within this proposed botnet detection framework. A primary concern is that the data collection methods employed by \citet{Moon2012} are only achievable through integration with an Internet Service Provider (ISP) infrastructure. Thus, it isn't easy to reproduce the same data collection process without internal access to an ISP service.

Similarly, the amount and type of information sourced from the KORNET ISP service varied significantly in scope and relevance to botnet propagation. The vast majority of binaries analysed were sourced from spam emails and malicious websites reported by KORNET users, with a significantly smaller number of binaries collected through botnet honeypots. The analysis of binaries collected through a combination of extraneous data-sources led to a trivial detection accuracy of 40\%, stipulating that a more refined data-collection process is required.

As botnets have evolved to target vulnerable IoT devices, conventional honeypot detection systems, such as Honeyd \citep{Honeyd2008}, have proven ineffective in identifying and monetising from IoT botnet traffic. The honeypot systems proposed by \citep{PaPa2016} and \citep{Antonakakis2017} can identify such traffic through configuring honeypot servers to mimic characteristics of commonly targeted IoT devices. Despite proving successful in collecting IoT malware, both implementations resulted in sparse data collection due to small-scale deployment and distribution of honeypots. The implementation proposed by \citep{PaPa2016} combines IoT honeypots and sandboxing environments using embedded Linux virtual machines, allowing for the purposeful infection of sandboxes and monitoring all incoming telnet based botnet communication. However, this system required frequent resets due to successful intrusions and only focused on monitoring telnet-based honeypot intrusions. By exclusively monitoring telnet traffic, this system overlooked important information provided by analysing other commonly used botnet communication protocols, including SSH, HTTP, P2P, and IRC traffic. As botnets frequently operate across various communication protocols, the exclusion of such protocols may lose important information valuable towards botnet identification.

\citet{Bastos2019} and \citet{Ceron2019} further build upon the concept of IoT honeypots and the execution and examination of malware within sandbox environments using a combination of static, dynamic, and heuristic analysis techniques. Both papers propose the separation of botnet data collection and analysis through extracting malware binaries and evaluating in/outgoing SSH and Telnet traffic within secure sandbox environments via the Detux \citep{Detux2016} Linux sandbox. Despite supporting the emulation of up to 5 CPU architectures, the Detux sandbox is limited in functionality and only allows for basic static and dynamic analysis. Detux also presents limitations for real-time malware analysis. It cannot be easily scaled or parallelised like that of competitor malware sandboxes, hindering the possibility of real-time static and dynamic malware analysis of honeypot traffic. Additionally, the long-term distribution of honeypots across 15 Brazilian states allowed for mass collection of IoT malware binaries. However, the lack of global honeypot distribution may result in collected data-sets being skewed by regionally dominant botnets. 

\citet{Dwyer2019} employs a solution that allows for accurate botnet detection without the requirement of malware execution and analysis of malware samples. Globally distributed across three continents, a honeypot system collected the IP addresses of all detected traffic emitting Mirai-like scanning activity. The host-names and DNS records of a sample of IP addresses were further collected and analysed by applying machine-learning classification models to various extracted features. This approach gave high botnet detection accuracy for Mirai botnets; however, as with most other centralised botnet detection systems, this solution cannot be applied to that of de-centralised botnet structures. As de-centralised botnets use P2P protocols to communicate instead of standard protocols such as IRC, HTTP and DNS, Peer-to-Peer botnets collectively communicate messages between connected peers rather than directly through a C2 server. The adopting Peer-to-Peer ifrastructure mitigates the use of centralised Command \& Control servers altogether, thereby lessening the incentive to employ DNS-based communication tactics common to centralised botnets. 

The findings presented by \citet{Herwig2019} illustrate that new de-centralised botnets are coordinated through interaction with de-centralised Peer-to-Peer Distributed Hash Tables (DHT). The various forms of shell-injection exploits commonly seen originating from propagating Hajime botnets are discussed, highlighting how shell commands such as \texttt{wget} and \texttt{tftp} are used to download and execute binaries. However, little detail is given regarding how payloads, obfuscated through command-line arguments and other methods, are extracted before sample analysis. As many malware samples are likely obfuscated using command-line tool arguments, such samples must be extracted to ensure botnet identification is maximised. It is proposed that the public keys used by Hajime botnets can be used as unique identifiers when monitoring botnet propagation, as IP addresses frequently change ownership. However, the context of IP address ownership may be relevant to botnet identification, as patterns can be identified to indicate prominent hosting services and Autonomous Systems that are affiliated with botnet activity. 

As IoT devices become more popular amongst households and businesses and new vulnerabilities are exposed, the importance of real-time identification and profiling of botnet traffic has significantly increased. Various Internet Service Providers, services, and government entities seek to mitigate the problem of botnet-based cyber-attacks; nevertheless, the continual evolution of botnets has made cyber-attack mitigation exceedingly more difficult. Such parties require quick and efficient ways to identify, visualise, and monitor current botnet activity, such that botnets can be observed and countered during propagation. The above botnet identification methods show high reliability and accuracy for IoT botnet detection. Yet, many of the proposed solutions are computationally expensive, difficult to deploy, or do not account for botnets' evolutionary factors, as attackers are persistently seeking to exploit new vulnerabilities, attack new target devices, and minimise detection whilst doing so. 

Given the limitations of previous botnet identification strategies, I propose VisiBot, an alternative botnet detection framework that focuses on real-time, automatic malware extraction and analysis. Using a globally distributed honeypot service, the VisiBot Processing System employs automatic extraction and execution of IoT botnet malware through sandbox and heuristic analysis techniques using scalable and distributable technologies. Through the synchronous processing of malicious traffic streams detected by a widely distributed IoT honeypot system and the combined use of various heterogeneous data-sources, VisiBot allows for real-time analysis of botnet propagation, geographic distribution, and density. By applying four heuristics based on the combined findings of \citet{Bastos2019} and \citet{Herwig2019}, the automated processing system employed by VisiBot constantly analyses and extracts knowledge from incoming botnet traffic, collected through a combination of static, dynamic and network-based analysis techniques. The resulting data collected during this process permits the identification and visualisation of presently propagating centralised and Peer-to-Peer IoT botnets. 