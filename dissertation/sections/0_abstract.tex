\begin{abstract}
The substantial demand for the Internet of Things (IoT) has led to the mass production of IoT devices susceptible to common security vulnerabilities, such as login credential brute-forcing and Remote Code Execution (RCE). Bad actors have begun to exploit such vulnerabilities in IoT devices to build powerful IoT botnets to coordinate large-scale distributed cyber-attacks. The detection and mitigation of such botnets is a high priority for authorities and cyber-threat intelligence specialists, as such cyber-attacks have proven to impact countless services, organisations, and government entities significantly. However, the identification of modern botnets, such as those targeting IoT devices, has proven challenging for authorities due to continually changing communication protocols, target devices, attack vectors, and obfuscation techniques.

In this paper, I propose VisiBot, a real-time, flexible, and fully automated botnet detection system capable of identifying centralised and de-centralised IoT botnets. Through the purposeful execution of malware samples from a globally distributed honeypot network, four identification heuristics are applied to a combination of static and dynamic analysis information to infer candidate Command \& Control servers and Peer-to-Peer networks.  Over a 35 day data-collection period, this system processed over 82,050 botnet events and successfully collected, executed, and analysed a total of 1,654 botnet malware samples. By combining novel detection heuristics with static, dynamic, and network analysis information extracted using automated IoT malware sandbox techniques, the proposed system successfully identified 1,303 candidate Command \& Control (C2) servers and 6,876 Peer-to-Peer Nodes. As all analysis is fully automated, this framework allows for real-time identification and visualisation of several botnet characteristics, including the geographic botnet density, Peer-to-Peer interactivity, and Autonomous System interactions with candidate C2 servers.
\end{abstract}